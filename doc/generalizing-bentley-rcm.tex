\documentclass{kluwer-mem-copyright}
\usepackage{cclicenses,pdfsync}
\begin{document}
\begin{article}
\begin{opening}
\title{The TransmissionLab Framework, Version 1\thanks{This work is licensed under the Creative Commons NonCommercial-Attribution-ShareAlike License version 3.0(\byncsa). 
To view a copy of this license, visit http://creativecommons.org or send a letter to 
Creative Commons, 559 Nathan Abbott Way, Stanford, California 94305, USA.}}
\subtitle{Generalizing the RandomCopyingModel for Cultural Transmission Research}

\runningauthor{M.E. Madsen}
\runningtitle{Generalizing the RCM}

\author{Mark E. \surname{Madsen}}  
\institute{Department of Anthropology, University of Washington \email{madsenm@u.washington.edu}}


\date{March 2007}

\begin{abstract}
Bentley et al. 2007 present a simple null model for imitation in cultural populations.  
This ``random copying model'' attempts to provide a simple generative model for the imitation of 
``neutral'' traits within a well-mixed population, with some innovation/invention rate for 
new traits, and stochastic loss of low-frequency traits by drift.  The current document 
outlines a series of generalizations and refactorings of the basic agent-based model,
suitable for additional
experimentation and analysis of cultural null models as well as various types of biased copying.   The results of this generalization process is \emph{TransmissionLab}, a framework for modeling cultural transmission in an agent-based manner.  
The ABM itself is written in Repast 3.x, a popular Java-based agent modeling toolkit with good 
documentation, available on open-source license terms for all common platforms.
\end{abstract}
\keywords{cultural transmission, cultural evolution, agent-based modeling, RePast 3.x}


\end{opening}

\section{Introduction}
In their 2007 article, ``Regular rates of popular culture change reflect random copying,'' Alex Bentley, 
Carl Lipo, Harold Herzog, and Matthew Hahn present a simple model of random cultural imitation in a well-mixed
population.  This model posits a null hypothesis for the cultural transmission of ``neutral'' traits, which 
is that nothing more complex than random copying of a fellow population member's trait is needed to 
account for the patterns seen at the population level.  


I began generalizing the Bentley et al. model because it is clean,
simple, and gave me a baseline model with known analytic form, as well as empirical behavior to use as a test comparison as I restructured the software itself.   Starting from scratch allows
one to design without history or constraints, but it is often better to start
from something which works, and modify it incrementally, in order to understand whether model behavior is faithful to underlying theoretical models, rather than simply an artifact of the development process.  I chose to start with Bentley's RCM as my baseline, and have evolved it considerably (as of version 1.2).  The result is a fairly generic framework for running 
simulation models of cultural transmission, called \emph{TransmissionLab}.  

In the next several sections I outline the reorganization, refactoring, and
generalization of the simulation codebase I'm performing.  This document is a work-in-progress, and will be updated as we continue to develop and use the transmission framework for new research projects.  In this document, I focus not on the scientific aspects of transmission models, but upon the software engineering
aspects of \emph{TransmissionLab}.  The framework is written in Java, using the Repast agent-based simulation toolkit.  With suitable generalization and refactoring, \emph{TransmissionLab} will serve as a useful ``platform'' model
for both extending Bentley et al.'s analysis, but performing a wide variety of computational ``experiments'' of cultural transmission. 
 
\section{Modeling Philosophy}
One of the hardest aspects of using agent-based models in scientific research is
translating the theoretical ``model'' into the simulation realm.  In my past experience with simulations
written in Swarm (Objective C), Repast (Java), and from scratch in C++, minor
``bugs'' are the least of one's worries.  Even a relatively simple simulation is a complex body
of code, and ensuring the correct timing and ordering of agent updates,
interactions, and observations in order to measure what we \emph{believe} we're
measuring can be quite difficult.  

In developing TransmissionLab, I began with the hypothesis that 
in most agent-based models designed to study
population-level ``inheritance'' problems, several aspects of the model can be
kept strictly orthogonal in the design of the simulation model.  And if the code
can be kept as separate as possible, each ``facet'' of such a model can be
tested rigorously with simple, ``null'' cases.  Schedules and event interleaving
can be tested without complex activity getting in the way.  Data collection,
displays and analysis can be tested on known data sets.  Agent interaction can
be tested in isolation, knowing that data collection or other aspects of the
model aren't affecting that interaction via side effects.  

Of course, the price of all this cleanliness and separation is a more complex
body of code.  What follows are notes on the current version of TransmissionLab, and the changes I'm trying in order to achieve a model which
really lives up to the promises of the previous paragraph.  These notes will
also serve as partial documentation for using and extending the model once
released.  

\section{Basic Reorganization and Runtime Updates}
For ease of constructing many models from the basic codebase, I reorganized the
model into a series of java packages under a ``src'' directory.  Packages for
this model are located under \texttt{org.mmadsen.sim.transmission} and comprise
a functional decomposition of the code:\\
\begin{tabular}{|r|r|}
\hline
Package&Function\\
\hline
agent&Classes implementing alternative types of agents\\
analysis&Classes implementing data collection or analysis upon a model\\
interfaces&Generic interfaces general model interrelationships\\
models&Classes which extend \texttt{SimModelImpl} and define a given model\\
population&Classes which implement agent populations\\ 
rules&Classes which implement transmission rules\\
test&JUnit test classes and test harnesses \\
util&Helper and utility classes\\
\hline
\end{tabular}


\section{Refactoring Notes}
At a high level, our goal is to create a framework for examining the dynamics of
many different cultural transmission models and scenarios, with agents of
varying internal complexity, different types of adoption and imitation rules
(e.g., random, imitate most frequent, imitate the successful), and differing
population structure (e.g., well-mixed, lattices, and social network graphs of
varying topology).  

We also seek an easy way to perform experimental measurements of these dynamics,
at both individual and population levels.  The platform should allow new types
of measurements to be ``plugged in'' to the basic model framework, and given
only knowledge of agent and model public methods, the plug-in should manage its
own data collection, statistical analysis, displays and graphs, and data
persistence.  Code for data collection and visualization should not be mingled
with model code, and the former must use clean public APIs to access the latter.
This philosophy promotes modularity, simple extension of existing transmission
models to look at new measurements and types of analysis, and helps ensure the
stability and accuracy of the model core once it is stable and the bug count is
low. 
\subsection{Refactoring Model and Measurement}
The first separation I performed was to extract data collection code from
the core model class (named \texttt{TransmissionLabModel}).  The current design
defines a single interface for measurement and data collection classes (named
\texttt{IDataCollector}).  This interface defines a simple four-method contract
which defines a data collection ``cycle.''  

In addition, each \texttt{IDataCollector} class carries a ``type code,'' which
allows the model to keep a map of which data collectors are running by type, and
easily access them at any time.  This is useful for removing or disabling data
collection based upon GUI settings, or altering the schedule for data collection
at runtime.  For example, we might want data snapshots of the running model
gathered, but only at specific points in the simulation run, and we don't want
to pay the performance penalty of scheduling the snapshot code when it's not
needed.  In the present version of the model, the ``type code'' for an
\texttt{IDataCollector} instance is the \texttt{getClass().getSimpleName()}, and
does not currently distinguish between identically named classes in different
packages or between object instances of the same class, if they perform
different data collection functions.  This needs to be evolved to guard against
such issues.

The model class stores a 
\texttt{List<IDataCollector>}, to which all data collection objects should be
added.  The simulation author, within the Repast \texttt{setup()} method, 
should instantiate any \texttt{IDataCollector} classes, call
their \texttt{build()} methods (to allow setup and reset to occur), and add the
objects to the data collector list.  The objects should also be added to a
\texttt{Map<String, IDataCollector>} which stores the data collector instances
under the ``type code'' key discussed above.  The simulation author is also
responsible for inserting the object instances into the data collector map using
\texttt{getDataCollectorTypeCode()} to retrieve the collector object's type code. 

The model class then takes care of initialization, in the Repast
 \texttt{begin()} method, by iterating over the data collector list, calling
\texttt{initialize()} on each data collection object.  We separate object
construction and initialization because Repast does so within models, so that
the simulation can run and display the basic GUI to gather parameters,
which are then available by the time the user clicks the ``begin'' button. 
Thus, in the \texttt{begin()} model method, we allow each data collector to
query the model for parameters relevant to its own configuration and operation.

[[DEPRECATED:  Describe the new scheduling mechanism]]
In the current version, scheduling of data collection is very simplistic, and
not well refactored.  Currently, the model's Action methods (\texttt{mainAction}
and \texttt{initialAction}) each iterate over the list, calling the
\texttt{process()} method.  If a given data collector is not supposed to run on
each model tick, as with data file recording in the original Random Copying
Model, the data collector itself is responsible for detecting this and returning
from a call to \texttt{process()} without performing any action.  

This is clearly not a clean design; ultimately each data collector object will
be able to add itself to a schedule which will call \texttt{process()} at the
correct times and intervals, and the current list iteration will be removed from
the Action methods. (Note:  as of version 1.2, the data collector objects are
added to an ActionGroup which is scheduled, with a BasicAction method for each
\texttt{process()} method.  This still doesn't handle more complex scheduling
details, but I'm going to leave that aside in 1.3 and work on modularity of 
transmission and other ``model'' rules in preparation for spatial structure.  

\subsection{Inter-Module Data Sharing}


\subsection{Data Collector Refactorings in the Bentley Model}
Given this structure, the turnover graph and data snapshots in the original
Random
Copying Model were moved into separate classes in TransmissionLab:\\  \texttt{top40DataFileRecorder}
and \texttt{TurnoverGraphCollector}.  The latter refactoring, in particular,
removes a great deal of code and complexity from the model class. 
The inner class \texttt{Turnover} moves to the \texttt{TurnoverGraphCollector},
and at some point the calculation of sorted ``top40'' lists will as well, since
the latter constitute \emph{analysis} of a transmission model (random copying)
rather than part of the agent interaction itself.

As part of this refactoring, I discovered that the original methods of
calculating turnover within the model fail under certain circumstances.  This
isn't an issue for Bentley et al. 2007, because the data processing for that
article was performed outside the Repast context (Bentley, personal
communication).  But in extending the analysis and models to more complex cases,
it would be good to have analysis modules which could reduce the amount of data
post-processing required.  

One ``programmers'' note about the refactorings described here may help others
extend the model later. It's taken a bit of work to
get everything working correctly and cleaning up after itself, mostly because
Repast is still a fairly low-level modeling framework (Repast Simphony promises
to change this dramatically). Repast, like
Swarm, isn't particularly good about separating initial invocation of the model
and subsequent runs of the model from the same invocation (e.g., by hitting the
stop and reset buttons in the GUI, or multiple batches, etc).  In particular, by
putting the graph and its collection class into a separate object, I ended up
with double construction and then null pointer errors unless the
\texttt{setup()} method in the model kept good track of whether we'd already
been through the loop or not.  This is the origin of the \texttt{completion()}
method in the \texttt{IDataCollector} interface:  it allows each data collector
to clean itself up, dispose of GUI windows and resources, close file references,
either at the end of a simulation or when the simulation is reset for a
subsequent run.  It's all working nicely now with no apparent memory leakage,
but the mechanics aren't very pretty.  Fortunately the code to track this sort of
thing is in the Model class, and Agents, IDataCollectors, or future simulation
``plug-in'' types will have to know anything about it.


\subsubsection{Example Data Collector:  TraitFrequencyAnalyzer}
[[NOTE: Update given changes to scheduling, etc]]

I added \texttt{TraitFrequencyAnalyzer}, written from scratch as a
``paradigm'' example of how a modular data collection and analysis system would
work.  The class accesses \emph{only} the agent list held by the model, caches
its own ``top N'' lists between model ticks, holds its own graph instances, and
performs all turnover calculations.  Thus, it can be ``turned off'' cleanly if
desired, and it does not inject any code into the model itself.  I describe this
class in some detail in this section, as a guide to writing additional data
collection and analysis modules in the future.  

Little initialization is required in the \texttt{build()} method for this class;
we instantiate a map for frequencies that will hold instances of an inner value
class (\texttt{TraitCount}), indexed by the agent's trait identifier (here, an
integer).  The \texttt{completion()} method is similarly very simple,
responsible only for ensuring disposal of any graphs or other GUI elements.  The
\texttt{initialize()} method similarly does little work, in this case
constructing two \texttt{OpenSequenceGraph}s when the user hits the ``begin''
button on a simulation.  One graph displays ``top N'' turnover over time, and
the second displays the total number of variants present in the population
throughout the simulation (with the pure random copying model and no additional
rules or structure, this converges on a mean value of 4Nmu), but since the
population is finite the value fluctuates around this infinite-limit value.

All action in this data collector is driven by the \texttt{process()} method,
called once per model tick by the model's scheduled ``main action.''  The first
task is to refresh our reference to the model's agent list, since we can make no
assumptions that it hasn't been altered (agents removed or added) since the
previous tick.  We then clear out the internal frequency map to count traits
afresh, and clear out an internal list of \texttt{TraitCount} objects which will
be filled later.  

Trait counting is done in a single pass through the agent list, using the Java
equivalent of a functional programming style.  By this I mean that we iterate
over the list, and apply a ``functor'' (function object) to each element.  This
is very similar to the style of programming familiar from Perl, Ruby, or Lisp
and is very efficient and clean.  This is done in (current versions of) Java by
creating an inner helper class that implements the \texttt{Closure} interface
from Jakarta Commons Collections, and then passing the agent list and the
closure class to the Commons Collections helper method
\texttt{CollectionUtils.forAlldo()}.  The closure class here is
\texttt{FrequencyCounter}, and in its \texttt{execute()} method it checks if the
passed \texttt{IAgent}'s trait is already indexed in the trait frequency map,
calls the \texttt{TraitCount.increment()} method if so, and if not, constructs a
new \texttt{TraitCount} object for the trait and inserts it into the map.

Once the counting pass is complete, the frequency map contains
\texttt{TraitCount} objects representing the frequency of each trait in the
agent population.  We then obtain a reverse-sorted list of these traits by
frequency by having \texttt{TraitCount} implement the \texttt{Comparable}
interface and define a \texttt{compareTo()} method which provides a ``natural''
sort order based on the count, not the trait ID or object hash code, and
reverses the ordering to obtain a descending order sort.  Given this comparator
method, the standard Java \texttt{Collections.sort()} method returns a list
already sorted with highest frequency first, lowest last.  This seems like a lot
of machinery to do a sorting, but we essentially get a ``top N'' list for free,
by simply reading the first N items off this sorted list.

To facilitate calculating turnover, we keep two versions of this sorted
\texttt{TraitCount} list:  a cached copy from the previous model tick and the
list being constructed for this model tick by \texttt{process()}.  Presumably we
could extend this by keeping the data from all model ticks if desired, or
persisting it to a database for analysis.  

Finally, the \texttt{process()} method calls the \texttt{step()} method on the
``turnover'' and ''variability'' graphs. The graphs
themselves are responsible for actually calculating turnover, much as in the original
model.  I use the same structure as the original model:  a
\texttt{TurnoverSequence} class which provides the graph with the latest
turnover value at each \texttt{step()} of the graph; similiarly, a
\texttt{TotalVariabilitySequence} provides that graph with a stream of
\texttt{size()} values for the current set of sorted TraitCount objects.

I calculate turnover slightly differently than the original model, in an attempt
to avoid those situations where there are less than 40 traits in the population
(which caused anomalies given fixed arrays since random data were being
incorporated into the analysis if the array wasn't filled all the way with
``real'' data).  In this class, I define:  \emph{turnover is the number of
elements in two sets (previous top N and current top N) that are not present in
the intersection of the two sets.}.  This definition counts traits that are
present in the previous list, but not present in the current list, as well as
the reverse situation.  The equation is thus:  
\texttt{turnover = (prevSortedTraitCounts.size() + curSortedTraitCounts.size())
- ( 2 * intersection.size());}.  Intersection is performed on two temporary
lists of the pure sorted trait IDs, passed to the Commons Collections utility
method\\ \texttt{CollectionUtils.intersection()}.  

If the lists are larger than 40 (really, a configurable parameter), we trim the
bottom of trait lists before performing the turnover calculation.  This means
that all of the original frequency counts are available to other methods, and
``top N'' restriction is only meaningful to this particular sequence class and graph.


\subsection{Plug-In Parameterization}
One thing that I'm still thinking through is how to deal with the Repast ``model
parameter'' subsystem and the notion of a plug-in architecture.  The difficulty
here is that the plug-in class (e.g., a class which implements
\texttt{IDataCollector}) ought to contain its own parameterization, which it
then contributes to the model at setup and model start.  Right now, things are a
bit of a hybrid:  \texttt{getInitParams()} is called so early in initialization
that my design for constructing plug-ins hasn't executed yet.  So I'm tracing
out how \texttt{getInitParams()} is called by \texttt{SimInit} and dependent
classes for initialization, to see what we can do.  

My ideal design would be to have some way to say ``run Model A, and do it with
analysis plug-ins X and Y.''  If I can't get there, you'll probably still have
to do some explicit loading and initialization of plug-in classes, but we could
perhaps use parameter files to let each plug-in class contribute parameters that
appear in the initial GUI.  

NOTE:  Still working on this (2/28/07) but am having some luck - in the
SimModelImpl constructor you can add PropertyDescriptors dynamically, like I do
for the initial trait structure pull-down.

\subsection{Refactoring Model and Inter-Agent Interaction}

\subsection{Isolating Population Structure}
Ultimately, my goal for agent populations is to allow other aspects of the model
(e.g., interaction rules, data analysis) to be orthogonal from the way agents
are arrayed in a population.  This will allow models to be held ``constant'' and
their results compared when the population is differently structured.  For
example, one might want to compare a well-mixed population baseline against
structures like regular lattices or various types of network models.  

With this goal in mind, I defined an interface \texttt{IAgentPopulation} which
represents an abstract agent population.  Each instance of
\texttt{IAgentPopulation} is created by a ``factory'' class, which must
implement the contract specified in \texttt{IPopulationFactory}.  The point of
this double abstraction is that different types of populations might require
different logic in their factory classes for construction, and creating small
units of encapsulation helps us avoid buggy, hard-to-understand factory classes.
 Since the factory classes all follow the same ``contract,'' however, we can
 configure a model class easily to load any factory and thus potentially
 construct any type of population.  One way this can be made dynamic is for each
 factory to export an XML descriptor of the population classes it is capable of
 constructing, perhaps by using XDoclet in the build process.  Then, in the
 model constructor (which runs prior to the parameter panel being constructed),
 the XML descriptors can be read in, and a pull-down list added to the initial
 model parameters.  Given the chosen class, in the model \texttt{begin()}
 method, we can dynamically load the correct factory class and call its
 \texttt{generatePopulation()} method.  
 
 NOTE:  As of 3/1/2007, I'm still working on this particular architecture.  At
 the moment the correct \texttt{IPopulationFactory} is referenced statically in
 the model class, and the population types are hardcoded into the model
 constructor.  This is very undesirable for clean expansion of the codebase, but
 I'm in transition as of the current version.  

\subsection{Cleaning Up Model Classes}
In creating a modular simulation template, it is turning out that the main model
class is both turning largely into boilerplate code and gaining some critical
internal complexity to handle the generic nature of the other simulation
structures.  At the same time, the model class itself does represent a key piece
of code for the modeler---containing simulation-level parameters, data
structures, and so on.  The modeler needs to construct a schedule for the
simulation, and instantiate.  

Solution here is to probably subclass all my generic mechanisms from
\texttt{SimModelImpl} and have our models subclass from there, supplying a
schedule, parameters, and any other model-level data fields.

\subsection{Miscellaneous Notes}
Several problems still exist that I need to work on, in addition to the
generalization work itself.  
\begin{enumerate}
  \item The fixed-size arrays that remain in the model class are somewhat
  fragile depending upon the combination of maxVariants and numNodes, now that
  the population is constructed by the new factory pattern, which knows nothing
  about ``maxVariants.''  I've tried to work on this by having
  \texttt{IAgentPopulation} classes track the ``largest'' variant they create
  during construction, and making that available to models via
  \texttt{getCurrentMaximumVariant()}.  Then, in the \texttt{mutateVariants()} 
  method, we increment the current notion of the ``biggest'' variant.  This is
  then used to re-allocate the various ``top40'' arrays in the original model
  methods.  The unfortunate thing is that I have a bug on the first time step if
  numNodes \( > \) initial MaxVariants.  At the moment the work-around is to
  ensure
  that the initial MaxVariants value is high enough, but I'll fix this bug soon.
\end{enumerate}

\acknowledgements
Thanks to Alex Bentley for providing the code from their 2007 paper and agreeing
to collaboration on a new generation of the model, and allowing free
distribution for others to use as well.  
\end{article}
\end{document}