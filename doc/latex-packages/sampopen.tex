\documentclass{kluwer}
\begin{document}
\begin{article}
\begin{opening}
\title{The `Kluwer' LaTeX Style File:\\
       An example \texttt{opening} environment\thanks{Donald Knuth}}            
\subtitle{Basic Instructions}

\author{A. \surname{Thor1}\email{thor@wkap.nl}}
\author{B. \surname{Thor2}\thanks{Partially support...}} 
\institute{KAP}                               
\author{C. \surname{Thor3}}
\institute{}

%\date: rather not

\dedication{To Jim}

\translation{De Kluwer LaTeX stylefile; aanwijzingen voor auteurs}

\runningtitle{De Kluwer LaTeX stylefile}
\runningauthor{Thor1 e.a.}

\begin{ao}
Kluwer Prepress Department\\
P.O. Box 990\\
3300 AZ Dordrecht\\
The Netherlands
\end{ao} 

\begin{motto}
What can't be done with TeX isn't worth doing.
\end{motto}

\begin{abstract} 
This document describes how to format a paper with
\texttt{kluwer.cls}, the Kluwer
classfile for journal submissions.
\end{abstract}

\keywords{LaTeX, Kluwer, Journal submissions}

\abbreviations{\abbrev{KAP}{Kluwer Academic Publishers};
   \abbrev{compuscript}{Electronically submitted article}}

\nomenclature{\nomen{KAP}{Kluwer Academic Publishers};
   \nomen{compuscript}{Electronically submitted article}}

\classification{JEL codes}{D24, L60, 047}
\end{opening}
\section{A heading}
Some text.
\newpage
\section{A heading}
Some text.
\newpage
\section{A heading}
Some text.
\newpage
\end{article}
\end{document}
