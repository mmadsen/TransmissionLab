% This document requires LaTeX2e.
% If you have trouble compiling this document, please use
% the included PostScript version, or download the pdf version.
%
\documentclass[namedreferences]{kluwer}

\usepackage{klups}
%   klups.sty selects Times as text font. If this document doesn't
%   compile, you may comment out this line.

\usepackage[dvips]{graphicx}
% if you use a different dvi-driver, the graphics in this document
% may not print.

\setcounter{tocdepth}{1}
\marginparwidth=0pt
\vbadness=5000
%\ifdef\textcopyright\else\let\textcopyright=\copyright \fi
\hfuzz=30pt
\parindent 2em
\def\Bs{\char92}
\def\BibTeX{\textsc{Bib}\TeX}
\usepackage{makeidx}
\def\indexname{Index}
\makeindex
\newcommand{\Pindex}[1]{\index{Packages!#1}\index{#1 package}}
\newcommand{\Eindex}[1]{\index{Environments!#1}\index{#1 environment}}
\newcommand{\Oindex}[1]{\index{Options!#1}\index{#1 option}}
\newcommand{\Cindex}[1]{\index{#1@\texttt{\Bs #1}}}
\newcommand{\dbr}{\discretionary{}{}{}}
\begin{document}
%\typeout{^^J*** Kluwer Academic Publishers - prepress department^^J***
%    Documentation for \filename, dated \filedate, version \fileversion}
\begin{article}


\begin{opening}
\title{User manual for {\tt kluwer.cls}}
\subtitle{Instructions for authors}
\author{Prepress \surname{Deparment}\email{texhelp@wkap.nl}\thanks{This
    e-mail address is available for all problems and questions.}}
\runningauthor{Kluwer Academic Publishers}
\runningtitle{User Manual}
\institute{Kluwer Academic Publishers, Achterom 119,
    3311 KB Dordrecht, The Netherlands}
\begin{ao}\\
KLUWER~ACADEMIC~PUBLISHERS PrePress Department,\\
P.O.~Box~17, 3300~AA~~Dordrecht, The~Netherlands\\
e-mail: TEXHELP@WKAP.NL\\
Fax: +31 78 6392500
\end{ao}
\date{\filedate}


\begin{abstract}
This is the user manual for the {\tt kluwer.cls} class file for the
preparation of articles to be submitted to journals.  It is based on
the same set of macros as the class files which Kluwer uses for the
typesetting of its journals.  The {\tt kluwer.cls} class file is
similar in use as the {\tt article} class file of \LaTeXe, but it
has some extra fields in the preamble and some extended commands
for the main text.
\end{abstract}
%
%
\keywords{\LaTeXe, (electronic) submission, user manual, class file, Kluwer}
\abbreviations{\abbrev{KAP}{Kluwer Academic Publishers};
                \abbrev{compuscript}{Electronically submitted article}}

\end{opening}

{\footnotesize\tableofcontents}

%%%%%%%%%%%%%%%%%%%%%%%%%%%%%%%%%%%%%%%%%%%%%%%%%%%%%%%%%%%%%%%%%%%%

\section{Introduction}

The {\tt kluwer} class file is the general style file for submitting
an article in \LaTeXe\ for publication in one of the journals of
Kluwer Academic Publishers.

The \LaTeX\ class files for typesetting Kluwer journals are based on
a common set of underlying macros; the same macros underly this
class file. So, in theory at least, conversion of your submission to
a journal class file should be trivial.

\subsection{Using other packages}

Please try to limit your use of additional packages. They frequently
introduce incompatibilities. This problem is not specific to
the Kluwer styles, it is a general \LaTeX\ problem.  Check this
manual whether the required functionality is already provided by
the Kluwer class file.  If you do need third-party packages, send them
along with the paper.  Also check the log file whether additional files
and packages are loaded by those packages and send them too.

%%%%%%%%%%%%%%%%%%%%%%%%%%%%%%%%%%%%%%%%%%%%%%%%%%%%%%%%%%%%%%%%%%%%

\section {The opening}

Although you may be able to include the body of your paper with few
or no
changes, the opening is specific to the Kluwer class files. This is
the structure of a paper:

{\small
\begin{verbatim}
\documentclass[options]{kluwer}

<package declarations>
<your own macros and preamble commands>

\begin{document}
\begin{article}

\begin{opening}
<opening commands>
\end{opening}

<body of your paper>

\end{article}
\end{document}
\end{verbatim}
}
It is possible to include several \texttt{article}
environments into one paper. For an edited volume, one can also
place a \texttt{frontmatter} environment before the \texttt{article}
environments.

\medskip\noindent
The opening environment should look like this:

{\small
\begin{verbatim}
\begin{opening}
\end{verbatim}
\noindent \% Title block:
\begin{verbatim}
\title{...}
\subtitle{...}
\end{verbatim}
\noindent \% Author block [repeatable]:
\begin{verbatim}
\author{...\surname{...}\email{...}}
\author{...\surname{...}\email{...}}
\institute{...}
\end{verbatim}
\noindent \% Extra information:
\begin{verbatim}
\dedication{...}
\translation{...}
\runningtitle{...}
\runningauthor{...}
\begin{ao}
....
\end{ao}
\begin{motto}
....
\end{motto}
\end{verbatim}
\noindent \% Abstracts and keywords:
\begin{verbatim}
\begin{abstract}
...
\end{abstract}
\keywords{...}
\abbreviations{\abbrev{...}{...};
               \abbrev{...}{...}}
\nomenclature{\nomen{...}{...}; \nomen{...}{...}}
\classification{...}{...}
\end{opening}
\end{verbatim}
\par} \noindent
These commands should be used only within the \texttt{opening} environment.
Most of them will produce an error if used somewhere else in your
document.  Below, we shall describe them in detail.  Their order
is not fixed. They are
executed by \verb+\end{opening}+, without regard to their location within
the \texttt{opening} environment (with an exception for the author block,
please see below for details). Anything that is not within a command
will end up being typeset {\it before\/} the `real' contents of the
environment, so be careful to only use the commands given here.

Now let us explain the commands in more detail.

\subsection{Title and Subtitle}

\Cindex{title}
\Cindex{subtitle}
The title and, if necessary, subtitle should be typed in one of
the following two styles of capitalisation:

\begin{trivlist}
\item[] {\tt \Bs title\{This is my title\}} \qquad \rm One initial capital \\
{\tt \Bs title\{This Is My Title\}} \qquad \rm Initial capitals
\end{trivlist}

\noindent
This has to be done according to the style of the journal in which your
article is to be published. Uppercasing will be done by the stylefile if
necessary. A missing \texttt{\Bs title} command generates an error message.

\Cindex{\Bs}You can split the typeset title and subtitle over multiple lines
by putting two backslashes at an appropriate place,
\Cindex{thanks}\index{Footnotes!to title}
and you can specify footnotes  to the title by use of the
\texttt{\Bs thanks}
command. The specified text will be placed at the bottom of the
page as a footnote.

An example of a title block could look like this:
\begin{verbatim}
\title{The `Kluwer' LaTeX Style File:\\
       Instructions for Authors\thanks{Donald Knuth}}
\subtitle{Basic Instructions}
\end{verbatim}

\subsection{Author block}

\Cindex{author}
\Cindex{institute}
The author block consists of one or more authors followed
by an institute. This construction may be used more than one time.
A typical `large' author block would look like this:
\begin{verbatim}
\author{A. \surname{Thor1}\email{thor@wkap.nl}}
\author{B. \surname{Thor2}\thanks{Partially supported...}}
\institute{KAP}
\author{C. \surname{Thor3}}
\institute{}
\end{verbatim}

\noindent \Cindex{surname} As you can see, the surname of the author
is given within a separate command, \texttt{\Bs surname}. This makes
it easier for us to extract database information from the \LaTeX\
file. It has no influence on the way the author's name is typeset.

\Cindex{thanks}\index{Footnotes!to author}
You can specify support acknowledgements etc. by use of the
\texttt{\Bs thanks} command.

\Cindex{email}
If the author has an email address that looks like it will last
for at least the time between submission and publication, you
can give the address in the \verb+\email+ command either after the
surname of the author or in the \verb+\institute+ command.
It will be typeset in
small typewriter font within brackets.

The specification for the author block is a bit more complicated
than the rest of the opening environment. Please
take note of the following restrictions:
\begin{enumerate}
\item Order is important for all \verb+\author+ and \verb+\institute+ commands.
    They will be typeset in the same ordering as found in your
    input file.
\item \verb+\thanks+\Cindex{thanks\space(to author)} can only be used as
    the very last item in the \verb+\author+
    command. If you make a mistake, you will get a quite incomprehensible
    \TeX\ error message.
\item The list of authors {\it has} to end with an institute. If there
    are authors that are not part of any institute, give an empty command
    as demonstrated above. If you don't, that author's name will {\it not}
    be typeset.
\item \index{Overfull lines} Sometimes the class file may fail to find
    a good place to break a line containing multiple authors. If this happens,
    you can insert\Cindex{\Bs}
    two backslashes (\verb+\\+) yourself to force a line break
    after an author's name or email address (It
    will {\it not} work if the two backslashes are the
    first characters within \verb+\author+. In this
    case there will be a \LaTeX\ error {\tt`No line to end'}).
\end{enumerate}

\subsection{Extra information}

These commands give some extra information about the article.

\subsubsection{Dedicated and translated articles}

\Cindex{dedication}
\Cindex{translation}
You can use the commands \verb+\dedication+ and \verb+\translation+ to give
extra information about your article. The commands help by inserting
the correct amount of space above and below your text.

\subsubsection{Running heads}

{\sloppy \Cindex{runningtitle}
You can create a running title (used on odd pages)
by using the \verb+\runningtitle{}+ command, where you put
the (maybe shortened) title between the curly braces,
\Cindex{runningauthor}
and you can create a running author line (used on even pages) using the
analogous \verb+\runningauthor+ command.\par}

Both of these commands may be used in strange ways by \LaTeX, so try to
keep the information short enough to fit in the headline, and try to use
only plain text. For instance, don't try to force line breaks in the
running title, and don't insert \verb+\label+s.

\subsubsection{Address for correspondence}

\Eindex{ao}
\Eindex{article}
The environment \texttt{ao} can be used to give an address for correspondence
or offprints.
The contents of this environment will be typeset at the end of the
article environment
(an example is given in this document on page~\pageref{lastpage}). The
exact phrasing of the leading text depends on the stylefile, so it may
change later here at Kluwer.


\subsection{Abstracts and keywords}

\Eindex{abstract}
The \texttt{abstract} environment is used to give a short description of
the article.

\Cindex{keywords}
The \verb+\keywords+ command gives a list of keywords. These should be
separated with commas.

\Cindex{abbreviations}
\Cindex{abbrev}
The \verb+\abbreviations+ command allows to give a list of
abbreviations used. It defines a sub-command \verb+\abbrev+  with two
arguments, of which the first is  the abbreviation and the
second the meaning. The exact layout will once again depend
on the used stylefile.

Here is the example from  the beginning of this document:
\begin{verbatim}
\abbreviations{\abbrev{KAP}{Kluwer Academic Publishers};
   \abbrev{compuscript}{Electronically submitted article}}
\end{verbatim}

\Cindex{nomenclature}
\Cindex{nomen}\noindent
The command \verb+\nomenclature+ is closely related to \verb+\abbreviations+.
It takes one argument subdivided into \verb+\nomen+ parts. The above
example gives the correct syntax.  For both environments it
is legal to have text outside of \verb+\abbrev+ respectively
\verb+\nomen+, in which
case it will just be typeset as flushed text, and you have to provide
additional formatting yourself. This is discouraged.


\Cindex{classification}
The command \verb+\classification+ takes two arguments, the first
describing the type of the classification, the second the classification
proper. An example might be:

\begin{verbatim}
\classification{JEL codes}{D24, L60, 047}
\end{verbatim}


\subsection{Date}

\Cindex{date}
By using the \verb+\date+ command, it is possible to specify reception and
acception dates. Since this is something that the journal's
editor enters, the default value when no \verb+\date+ command is given
is to insert a template line that allows easy fill-in of this
information. You are advised not to use this command.


\subsection{Motto}

When you want to specify a piece of poetry or a quotation to
specify a motto to your article, you can use the \texttt{motto}\Eindex{motto}
environment. You can differentiate between prose and poetry.
If you are using prose, give an extra option to the environment:
\texttt{[prose]}

Here is an example:
\begin{verbatim}
\begin{motto}[prose]
The only thing that never looks right is a rule.
There is not in existence a page with a rule on it
that cannot be instantly and obviously improved
by taking the rule out.

\rightline{George Bernard Shaw, In {\it The Dolphin} (1940)}
\end{motto}
\end{verbatim}
\Cindex{rightline}

\subsection{Example}

An example of an elaborate \texttt{opening} environment:
{\small
\begin{verbatim}
\begin{opening}
\title{The `Kluwer' LaTeX Style File:\\
       Instructions for Authors\thanks{With thanks to Donald Knuth}}
\subtitle{Basic Instructions}

\author{A. \surname{Thor1}\email{thor@wkap.nl}}
\author{B. \surname{Thor2}\thanks{Partially supported...}}
\institute{KAP}
\author{C. \surname{Thor3}}
\institute{}

%\date: rather not

\dedication{To Jim}

\translation{De Kluwer LaTeX stylefile; aanwijzingen voor auteurs}

\runningtitle{De Kluwer LaTeX stylefile}
\runningauthor{Hoekwater e.a.}

\begin{ao}\\
Kluwer Prepress Department\\
P.O. Box 990\\
3300 AZ Dordrecht\\
The Netherlands
\end{ao}

\begin{motto}
What can't be done with TeX isn't worth doing.
\end{motto}

\begin{abstract}
This document describes how to format a paper with
\texttt{kluwer.cls}, the Kluwer
class file for journal submissions.
\end{abstract}

\keywords{LaTeX, Kluwer, Journal submissions}

\abbreviations{\abbrev{KAP}{Kluwer Academic Publishers};
   \abbrev{compuscript}{Electronically submitted article}}

\nomenclature{\nomen{KAP}{Kluwer Academic Publishers};
   \nomen{compuscript}{Electronically submitted article}}

\classification{JEL codes}{D24, L60, 047}
\end{opening}
\end{verbatim}
\par}
\noindent Here ends the description of the \texttt{opening} environment.

%%%%%%%%%%%%%%%%%%%%%%%%%%%%%%%%%%%%%%%%%%%%%%%%%%%%%%%%%%%%%%%%%%%%

\section{Mathematical constructs}

As already indicated, the Kluwer class file offers additional
functionality be\-yond what the standard LaTeX \texttt{article} style
provides.  This section gives an overview of the additional
mathematical constructs that are supported.

\subsection{Symbols}

The class file attempts to load two packages \Pindex{amssymb}
\texttt{amssymb} and \Pindex{wasysym}\texttt{wasysym}. These support
additional symbols.
There will be no error if they are not available.

All of the  commands for symbols from these packages will
of course be available. Additionally, there are
some commands defined in \texttt{klumac.sty}. A short overview
is given in Table~\ref{figsym}.\footnote{%
   Some of these commands actually use the AMS fonts.
   This means that if you do not have these fonts, the
   commands will fail with a `missing font..' error.}

Please see the documentation for \texttt{amssymb} and \texttt{wasysym}
for the additional symbols these packages provide (most of \texttt{amssymb} is
also documented in~\cite{goossens}).

\begin{table}
\small
\begin{tabular*}{\maxfloatwidth}{llll}\hline
$\I         $&\verb+$\I$+         & $\down{a}$    &\verb+$\down{a}$+  \\
$\Res{a}$     &\verb+$\Res{a}$+   & $\dres$       &\verb+$\dres$+     \\
$\bigo{a}$   &\verb+$\bigo{a}$+   & $\forces    $&\verb+$\forces$+    \\
$\concat    $&\verb+$\concat$+    & $\grlo      $&\verb+$\grlo$+      \\
$\dC$         &\verb+$\dC$+       & $\gsim      $&\verb+$\gsim$+      \\
$\dE$         &\verb+$\dE$+       & $\iduaal$     &\verb+$\iduaal$+   \\
$\dF        $&\verb+$\dF$+        & $\lh$         &\verb+$\lh$+       \\
$\dL        $&\verb+$\dL$+        & $\logr$       &\verb+$\logr$+     \\
$\dN        $&\verb+$\dN$+        & $\lsim$       &\verb+$\lsim$+     \\
$\dP$         &\verb+$\dP$+       & $\mapdown{a}$&\verb+$\mapdown{a}$+\\
$\dQ$         &\verb+$\dQ$+       & $\mapright{a}$&\verb+$\mapright{a}$+\\
$\dR$         &\verb+$\dR$+       & $\nequiv    $&\verb+$\nequiv$+    \\
$\dZ        $&\verb+$\dZ$+        & $\oast$       &\verb+$\oast$+   \\
$\depth     $&\verb+$\depth$+     & $\range     $&\verb+$\range$+     \\
$\diamond$    &\verb+$\diamond$+  & $\res$       &\verb+$\res$+       \\
$\dom$        &\verb+$\dom$+      & $\restrict  $&\verb+$\restrict$+  \\ \hline
\end{tabular*}
\caption[]{Additional math symbols made available through \texttt{kluwer.cls}}
\label{figsym}
\end{table}

\subsection{Subequation environment}

A subequation environment is available to create equations with
sub-numbering of the equation counter. It takes one (optional)
argument, that
specifies the way that the sub-counter should be typeset.

Possible options are:

\begin{trivlist}
\item\begin{TABULAR}{@{}lrl}
alph & (1a) & This is the default.\\
Alph & (1A) & Uppercase\\
arabic & (1.1) & Arabic numbers.
\end{TABULAR}\label{subeq}
\end{trivlist}

\noindent \emph{e.g.}

\begin{verbatim}
\begin{subequation}[alph]
\begin{equation}
E = mc^2
\end{equation}
\begin{equation}
x^2 + y^2 = 1
\end{equation}
\end{subequation}
\end{verbatim}
\label{sub}
producing
\begin{subequation}[alph]
\begin{equation}
E = mc^2
\end{equation}
\begin{equation}
x^2 + y^2 = 1
\end{equation}
\end{subequation}
You can also put an \texttt{eqnarray} environment inside a
\texttt{subequation} environment:
\begin{verbatim}
\begin{subequation}[arabic]
\begin{eqnarray}
E & = & mc^2 \\
x^2 + y^2 & = & 1
\end{eqnarray}
\end{subequation}
\end{verbatim}
producing
\begin{subequation}[arabic]
\begin{eqnarray}
E & = & mc^2 \\
x^2 + y^2 & = & 1
\end{eqnarray}
\end{subequation}

\subsection{Varequation environment}

\Eindex{varequation}
If nothing else fits your requirements, you can also `hard-code'
equation numbers of your own design with the \texttt{varequation}
environment. No \LaTeX\ counters are incremented. You may
want to use it as follows:

\begin{verbatim}
\begin{varequation}{3\ensuremath{\alpha}}
E = mc^2
\end{varequation}
\end{verbatim}
producing
\begin{varequation}{3\ensuremath{\alpha}}
E = mc^2
\end{varequation}
Note the use of  \Cindex{ensuremath}\verb+\ensuremath+, which may be
necessary when you use mathematical symbols inside the argument.
Otherwise your input may fail under some internal stylefiles (i.e.
those that use equation numbers on the left side of the page).

Note also that `hard' equation numbers require `hard' equation
references: \LaTeX's automatic cross-referencing is of no use here.

\subsection{Displayed sentences: theorems and such}

The {\tt kluwer} class file contains a predefined layout for theorems,
corollaries, etc. These environments have to  be defined with the help of
\LaTeX's \verb+\newtheorem+ command.
Here is an example:
\begin{verbatim}
\newtheorem{theoremdemo}{Demo Theorem}
\begin{theoremdemo}
This is a  \verb!\newtheorem!.
\end{theoremdemo}
\end{verbatim}
Results in:
\newtheorem{theoremdemo}{Demo Theorem}
\begin{theoremdemo}
This is a  \verb!\newtheorem!.
\end{theoremdemo}

\Cindex{newdisplay}\noindent
The class file also defines a \verb+\newdisplay+ command, which is
identical to \verb+\newtheorem+ in usage, but will give a different
layout.
Here is an example:
\begin{verbatim}
\newdisplay{displaydemo}{Demo Theorem}
\begin{displaydemo}
This is a  \verb!\newdisplay!.
\end{displaydemo}
\end{verbatim}
Results in:
\newdisplay{displaydemo}{Demo Theorem}
\begin{displaydemo}
This is a  \verb!\newdisplay!.
\end{displaydemo}

\noindent
For more information on the use of the
\verb+\newtheorem+ command, please refer to the \LaTeX{} manual.


\Cindex{newproof}
Also, the class defines an extra command, \verb+\newproof+, that can
be used for displayed text of a lower level. The major differences are
that this command does not define a counter to go with the environment,
nor does it have an option to specify such a counter.

\begin{verbatim}
\newproof{rem}{Remark}
\begin{rem}
This is a remark.
\end{rem}
\end{verbatim}
resulting in
\newproof{rem}{Remark}
\begin{rem}
This is a remark.
\end{rem}
\noindent except of course that you should put commands such as
\verb+\newproof+ in the preamble.

\subsection{Proofs}\Eindex{pf}
For proofs, you can use the \texttt{pf} environment. This will generate
the heading `{\it Proof.}' If you wish to create your own heading,
you can use \Eindex{pf*}{\tt pf*} and give the \verb+\begin{pf*}+ command
your alternative heading as an extra argument. Thus,
\verb+\begin{pf*}+\verb+{Proof (of Lemma 3)}+ will generate the heading
`{\it Proof (of Lemma~3).}'


%%%%%%%%%%%%%%%%%%%%%%%%%%%%%%%%%%%%%%%%%%%%%%%%%%%%%%%%%%%%%%%%%%%%


\section{Sectioning commands}\Cindex{section}
This section gives examples of sectioning
commands.

\section{Section}
Here is a section
\subsection{Subsection}\Cindex{subsection}
Here is a subsection
\subsubsection{Subsubsection}\Cindex{subsubsection}
And a subsubsection
\paragraph{Paragraph}\Cindex{paragraph}
This is a lead-in heading (paragraph)
\subparagraph{Subparagraph}\Cindex{subparagraph}
This is a lead-in heading (subparagraph)


\noindent
There is also a command \verb+\part+\Cindex{part}, but this is not
intended to be used within articles.

The general syntax of these commands should be known to you
already. We have made no changes apart from the layout.

%%%%%%%%%%%%%%%%%%%%%%%%%%%%%%%%%%%%%%%%%%%%%%%%%%%%%%%%%%%%%%%%%%%%

\section{Lists and displayed text}
The document class redefines \texttt{itemize}\Eindex{itemize} and
\texttt{enumerate}\Eindex{enumerate} so that they take an optional
argument that makes some parametrization possible.

Also, there are some environments available for displayed
text, for instance, quotations.

\subsection{itemize}

In the case of \texttt{itemize}\Eindex{itemize} the optional
argument specifies a default label:

\begin{verbatim}
\begin{itemize}[$\bullet$]
\item This is an example of a list with an optional argument
\item[--] This is the second item
\end{itemize}

\begin{itemize}
\item Ordinary.
\end{itemize}
\end{verbatim}

Will result in:
\begin{itemize}[$\bullet$]
\item This is an example of a list with an optional argument
\item[--] This is the second item
\end{itemize}


\subsection{enumerate}

In the case of \texttt{enumerate}\Eindex{enumerate} the optional
argument can be used to specify the width of the left margin, in
approximately the same way as for the\break \texttt{thebibliography}
environment.

\begin{verbatim}
\begin{enumerate}[000000]
\item See the wide left margin below?\\ Next line
\end{enumerate}
\end{verbatim}

\noindent Will result in:
\begin{enumerate}[000000]
\item See the wide left margin below?\\ Next line
\end{enumerate}

\begin{verbatim}
\begin{enumerate}
\item Normal.\\ Next line
\end{enumerate}
\end{verbatim}

\noindent Will result in:
\begin{enumerate}
\item Normal.\\ Next line
\end{enumerate}

\noindent The default value for the optional argument is \texttt{00},
which is approxi\-mately 1em.

\subsection{Using poetry, quotes, and quotations}

LaTeX defines environments
\texttt{verse}\Eindex{verse}, \texttt{quote}\Eindex{quote} and
\texttt{quotation}.\Eindex{quotation}; see \cite{lamport}.
Here is how they look with the Kluwer class file:

\noindent
Here's a verse:
\begin{verse}
I think you will like them,\\
when you shall see them on a beautiful quarto page,\\

{\sc richard brinsley sheridan}, %
  {\itshape The School for Scandal\/} (1777) % Act I Sc 1
\end{verse}

\noindent
A quote:
\begin{quote}
When a proof has been sent me with two or three
lines so widely spaced as to make a grey band across the page,
I have often rewritten the passage so as to fill up the lines better

{\sc george bernard shaw}, in {\itshape The Dolphin\/} (1940) % v4 p80
\end{quote}

\noindent
A quotation:
\begin{quotation}
There was things which he stretched,
but mainly he told the truth.

{\sc mark twain}, {\itshape Huckleberry Finn\/} (1884) % chap1
\end{quotation}

\subsection{Vitae for authors}

Because a lot of our articles contain
a short vita(e) of the author(s), we have provided a
\texttt{vitae}\Eindex{vitae}\Cindex{Vauthor}
environment. Usage is as follows:
\def\manai{{\slshape Author}}
\begin{verbatim}
\begin{vitae}
\Vauthor{Author1} ...
\Vauthor{Author2} ...
\end{vitae}
\end{verbatim}

\noindent This environment is smart enough to notice the difference
between one author and more than one author, which will be reflected
in the generated heading.
\nopagebreak
\begin{vitae}
\Vauthor{Hoekwater, T.} This author spent most of his working life
    answering questions about \TeX\ and \LaTeX.
\Vauthor{Kroonenberg, S.} Now this is the job of the second
    author.
\end{vitae}

\subsection{Special sections for notes and acknowledgements}

Just as most articles have a bibliography, a lot of articles
also contain `Notes' and `Acknowledgements'. For this reason,
there are respectively the environments \texttt{notes}\Eindex{notes}
and \texttt{acknowledgements}\Eindex{acknowledgements}.

Here are some examples again:
\begin{verbatim}
\begin{notes}
\item Please note that this class file is provided
    {\it as is}, and is copyrighted\index{copyright}
    by Kluwer Academic Publishers. You are free to use
    this class file as you see fit, provided that you do
    not make changes to the included macro files.
    If you do make changes, you are required to rename
    the changed files.\index{making changes}
\end{notes}
\end{verbatim}

\begin{notes}
\item Please note that this class file is provided {\it as is}, and
    is \textcopyright\index{copyright}
    by Kluwer Academic Publishers. You are free to use
    this class file as you see fit, provided that you do not make
    changes to the included macro files. If you do make changes, you
    are required to rename the changed files.\index{making changes}
\end{notes}

\begin{verbatim}
\begin{acknowledgements}
...
\end{acknowledgements}
\end{verbatim}

\begin{acknowledgements}
I would like to thank Donald E. Knuth for the fact that
he wrote this brilliant program, thereby indirectly supporting
my wife, my cats and myself.\quad({\it T.H.\/})
\end{acknowledgements}

%%%%%%%%%%%%%%%%%%%%%%%%%%%%%%%%%%%%%%%%%%%%%%%%%%%%%%%%%%%%%%%%%%%%

\section{Footnotes and endnotes}
\index{Footnotes}
\index{Endnotes}

When using the {\tt kluwer} class file, you can produce endnotes
analogous to \LaTeX\ footnotes. Instead of the
\verb+\footnote+\Cindex{footnote} command you use
\verb+\endnote+\Cindex{endnote}. The command
\verb+\theendnotes+\Cindex{theendnotes} should be used to place the
endnotes in the text. They will be put in a separate `Notes' section
in the \verb+\footnotesize+ font.

There are also companion commands \verb+\endnotemark+\Cindex{endnotemark}
and \verb+\endnotetext+\Cindex{endnotetext}. Use those under
circumstances when \verb+\endnote+ would fail.\endnote{Here is an example
of an endnote}

%%%%%%%%%%%%%%%%%%%%%%%%%%%%%%%%%%%%%%%%%%%%%%%%%%%%%%%%%%%%%%%%%%%%
\section{Kluwer's tabular environment}\label{tabular}

The Kluwer class file makes various
changes\Eindex{tabular}\index{tabular!changes made in}
\Eindex{TABULAR(*)} to the \texttt{tabular} and \texttt{tabular*}
environments. Most notable is the extra vertical spacing. This
effectively disables vertical rules, since they would contain gaps.
Vertical rules rarely help to make a table more readable, and are
frowned upon by designers. For most tables, the recommended
arrangement of rules is: \\
-- one rule at the top \\
-- one rule separating column heads from data cells \\
-- one bottom rule \\
and no vertical rules; see \emph{e.g.} table~\ref{tab1} in
section~\ref{floats}.
But if you think you really can't live without vertical rules, you
can use the \texttt{TABULAR}\index{TABULAR}\Eindex{TABULAR}
environment, which is identical to the standard LaTeX
\texttt{tabular} environment.  Section \ref{floats} contains
examples.

\subsubsection{tabular* and the array package}

\noindent The \texttt{tabular*} environment is also modified: it
always fills the entire specified width.\Eindex{tabular*}; see
table~\ref{tab2} in section~\ref{floats}.

If you load the \texttt{array} or \texttt{dcolumn}
package, then the \texttt{tabular*} environment won't spread its
columns automatically; you need to create the more elaborate
preamble described in Lamport C.10.2. Whereas \emph{without} the
array package a preamble
\begin{verbatim}
\begin{tabular*}{\maxfloatwidth}{lll}
\end{verbatim}
would suffice, you would need
\begin{verbatim}
\begin{tabular*}{\maxfloatwidth}{%
  l@{\extracolsep{0pc plus 2pc}}l@{\extracolsep{0pc plus 2pc}}l} 
\end{verbatim}
if the array- or dcolumn package is loaded.

\subsection{Additional commands for `tabular' and `tabular*'}
\index{tabular!additional commands in}

As you saw above, there are some changes to the \texttt{tabular}
and \texttt{tabular*} environments. Apart from the extended baseline
and the filling out of \texttt{tabular*}, there are also four extra
commands to draw partial horizontal lines; these are:

\medskip
\Cindex{rcline}\Cindex{lcline}\Cindex{rlcline}\Cindex{lrcline}%
\index{tabular!lines in}\noindent
\verb+\rcline+, which leaves some space to
the {\it left\/} of the line\\ \verb+\lcline+, which leaves space to
the {\it right\/} of the line\\ \verb+\rlcline+ and \verb+\lrcline+,
these two leave space on both sides (they mean the same
thing).
\medskip

\noindent These commands are used in the tabular below:


\bigskip\noindent
\begin{tabular}{lll} \hline
Description 1 & Description 2 & Description \\
\lcline{1-1}\rlcline{2-2}\rcline{3-3}
A& Row 1, Col 2 & Row 1, Col 3 \\
B& Row 2, Col 2 & Row 2, Col 3 \\ \hline
\end{tabular}

\begin{verbatim}
\begin{tabular}{lll} \hline
Description 1 & Description 2 & Description \\
\lcline{1-1}\rlcline{2-2}\rcline{3-3}
A& Row 1, Col 2 & Row 1, Col 3 \\
B& Row 2, Col 2 & Row 2, Col 3 \\ \hline
\end{tabular}
\end{verbatim}

%%%%%%%%%%%%%%%%%%%%%%%%%%%%%%%%%%%%%%%%%%%%%%%%%%%%%%%%%%%%%%%%%%%%

\section{Floats}\label{floats}
\index{floats}

{\bf Note.} When using the Kluwer float environments, keep in mind
that the Kluwer classfiles need to provide uniform formatting.
Whereas the standard LaTeX float environments simply provide a plain
vanilla floating environment with a caption command, the Kluwer
float environments contain various formatting specifications;
\emph{e.g.} the placement of the caption within a float is enforced:
the caption will be typeset above the table whether your caption
command preceeds or follows your tabular environment. If you run
into trouble, try to remove your own formatting and go with the
defaults provided.

\bigskip\noindent
The {\tt kluwer} class file defines two major kinds of float, tables
and figures, and some derivatives like starred environments and {\tt
sub...} versions.

\subsection{The table environment}
\Eindex{table}\index{Tables}
\noindent Here is how the \texttt{table} environment is intended to be used:
\begin{verbatim}
\begin{table}[placement options]
...
\caption[optional short version]{Caption text.}
\label{unique-label}
\end{table}
\end{verbatim}
\label{capverb}

\noindent For {\itshape placement options}\index{floats!placement
options}\index{Tables!placement options}\index{Figures!placement
options} you can use a combination of the following characters:
\begin{itemize}
\item[{\tt t}] This indicates placement at the top of a page. This may
    be the current page, but it could also be a subsequent one.
\item[{\tt b}] This indicates placement at the bottom of a page. This may
    be the current page, but it could also be a subsequent one.
\item[{\tt p}] This indicates placement on a page that contains floats only,
    no text.
\item[{\tt h}] This should mean `here', but it is not always possible
    (there may not be enough  space left on the page). In that case, \LaTeX\
    will change this to {\tt [ht]}.
\end{itemize}
\noindent
The following option is non-standard:\index{The {\tt H} option for floats}
\begin{itemize}
\item[{\tt H}]  Also means `here', but in a more rigorous way. If there
    is not enough space left on the page, the page will be filled up
    with whitespace and
    the float will appear at the top of the next page. The difference between
    {\tt h} and {\tt H} is that {\tt H} {\it never\/} skips over the following
    text.
\end{itemize}\label{specifiers}

\noindent The table \Cindex{caption}caption is given with the
\verb+\caption+ command. This command is required! If you do not
have a caption, use \verb+\caption{}+.\index{floats!captioning} As
you will see in the examples below, the correct way to tag a table
for referencing through \verb+\ref+ is by putting  the \verb+\label+
{\it after\/} the caption command.\index{floats!referencing}


\subsubsection{Examples}
We show two examples of tables. Their output can be seen in
Tables~\ref{tab1} and~\ref{tab2}. See section~\ref{tabular} on the
use of rules in tables.

The first example, table~\ref{tab1}:
\begin{verbatim}
\begin{table}
\caption{Caption text.}\label{tab1}
\begin{tabular}{lll} \hline
Description 1 & Description 2 & Description \\ \hline
Row 1, Col 1 & Row 1, Col 2 & Row 1, Col 3 \\
Row 2, Col 1 & Row 2, Col 2 & Row 2, Col 3 \\
\hline
\end{tabular}
\end{table}
\end{verbatim}


\begin{table}
\caption{Caption text.}\label{tab1}
\begin{tabular}{lll} \hline
Description 1 & Description 2 & Description \\ \hline
Row 1, Col 1 & Row 1, Col 2 & Row 1, Col 3 \\
Row 2, Col 1 & Row 2, Col 2 & Row 2, Col 3 \\ \hline
\end{tabular}
\end{table}

\noindent
The following example demonstrates the Kluwer version of the
\texttt{tabular*} environment; see section~\ref{tabular}. You may
want to set the width parameter to
\verb+\maxfloatwidth+,\Cindex{maxfloatwidth} \index{floats!maximum
width} which is the maximum allowed width of the float. Some of our
journals have indented floats, so then the float is not allowed to
be as wide as \verb+\textwidth+, while rotated tables (see below)
have an allowed width of (almost) \verb+\textheight+.

\begin{verbatim}
\begin{table}
\caption{Caption text.}\label{tab2}
\begin{tabular*}{\maxfloatwidth}{lll} \hline
Description 1 & Description 2 & Description \\ \hline
Row 1, Col 1 & Row 1, Col 2 & Row 1, Col 3 \\
Row 2, Col 1 & Row 2, Col 2 & Row 2, Col 3 \\
\hline
\end{tabular*}
\end{table}
\end{verbatim}

\begin{table}
\caption{Caption text.}\label{tab2}
\begin{tabular*}{\maxfloatwidth}{lll} \hline
Description 1 & Description 2 & Description \\ \hline
Row 1, Col 1 & Row 1, Col 2 & Row 1, Col 3 \\
Row 2, Col 1 & Row 2, Col 2 & Row 2, Col 3 \\
\hline
\end{tabular*}
\end{table}

\subsubsection{Trouble with captions}
\noindent One final remark on tables: a Kluwer table float is
specifically intended to contain a \texttt{tabular[*]} environment.
If it doesn't, it may not know how wide a caption should be. If a
long caption doesn't get broken properly into lines, the cause may
be that your table is not a tabular, or that you are using a package
that reimplements the \texttt{tabular} environment, or that there is
more than one \texttt{tabular}. In such cases, you can tell the
caption command manually the width of the table with a command
\verb+\tabwidth{...}+ at the end of the \texttt{table} environment.

\subsection{The figure environment}\index{Figures!options|see{floats}}

Section~\ref{graphics} deals with general aspects of including
graphics. This subsection is about the Kluwer
\texttt{figure} float environment.

For figures, there are the environments
\texttt{figure}\Eindex{figure} and \texttt{figure*}\Eindex{figure*}.
Both of these support the same optional placement specifiers
as \texttt{table};
see page~\pageref{specifiers} and section~\ref{specifiers}.

Figure~\ref{mouse} is a typical figure, generated by the following
code:
\begin{verbatim}
\begin{figure}
\centerline{\includegraphics[width=12pc]{mouse.eps}}
\caption{This is the caption of the figure.}
\end{figure}
\end{verbatim}
Note that this example needs the line
\Pindex{graphicx}
\begin{verbatim}
\usepackage[...]{graphicx}
\end{verbatim}
in the preamble of your document.


\Eindex{figure}\Cindex{caption}\Cindex{centerline}
\begin{figure}
\centerline{\includegraphics[width=12pc]{mouse.eps}}
\caption{This is the caption of the figure. This is a little mouse
with one ear.}\label{mouse}
\end{figure}


\subsection{Subnumbering tables and figures}

\texttt{figure}\Eindex{figure} and \texttt{table}\Eindex{table} also
allow you to use the environments \texttt{subtable}\Eindex{subtable}
and \texttt{subfigure}\Eindex{subfigure}. These two environments
implement subnumbering, analogous to the
\texttt{subequation}\Eindex{subequation} environment; see
section~\ref{sub}.\index{floats!subnumbering}

\subsection{Side-by-side figures}
Please don't use the \texttt{subfigure} package for placing figures
side-by-side: there is a name conflict with the subfigure feature
described above. You won't need any particular package for this
anyway. One way to accomplish such a thing is with the following
code; see figure~\ref{twomice}:
\begin{verbatim}
\begin{figure}[H]
\tabcapfont
\centerline{%
\begin{tabular}{c@{\hspace{6pc}}c}
\includegraphics[width=1in]{mouse.eps} &
\includegraphics[width=1in]{mouse.eps} \\
a.~~ A mouse & b.~~ Another mouse
\end{tabular}}
\caption{Two mice}\label{twomice}
\end{figure}
\noindent Figures \ref{twomice}.a. and \ref{twomice}.b. are
suspiciously similar.
\end{verbatim}

\begin{figure}[H]
\tabcapfont
\centerline{%
\begin{tabular}{c@{\hspace{6pc}}c}
\includegraphics[width=1in]{mouse.eps} &
\includegraphics[width=1in]{mouse.eps} \\
a.~~ A mouse & b.~~ Another mouse
\end{tabular}}
\caption{Two mice}\label{twomice}
\end{figure}
\noindent Figures \ref{twomice}.a. and \ref{twomice}.b. are
suspiciously similar.

\subsection{The algorithm environment}

The \texttt{algorithm} environment\Eindex{algorithm} is no longer
available.

\subsection{Rotated floats}
\index{floats!rotating}
\noindent To rotate a table or figure over $90^{\circ}$
counterclockwise, enclose the entire table/figure in the
\texttt{kaprotate}\Eindex{kaprotate} environment.

\begin{verbatim}
\begin{kaprotate}
\begin{table}
...
\end{table}
\end{kaprotate}
\end{verbatim}

\noindent If you rotate a float, it will always use an entire page.

\subsection{Single- and doublecolumns floats}
In two-column styles, there is a difference
between\index{floats!difference between normal and starred forms}
the starred forms \texttt{figure*} and \texttt{table*} compared to
their normal forms. The starred forms of \texttt{table} and
\texttt{figure} will be doublecolumn, while the ordinary forms are
one column.  This distinction also holds for rotated versions.

\section{Using graphics with LaTeX}\label{graphics}

\subsection{Macro packages for graphics inclusion}

You are encouraged to include figures electronically. For this, we
recommend the packages from the graphics bundle, including
\texttt{graphics}\Pindex{graphics},
\texttt{graphicx}\Pindex{graphicx} and
\texttt{epsfig}\Pindex{epsfig}. If you currently use
\texttt{psfig}\Pindex{psfig} then you may prefer \texttt{epsfig},
which implements the interface of \texttt{psfig} in terms of the
other packages. For documentation, see `Packages in the
`graphics' bundle' by David Carlisle (filename
\texttt{grfguide.[tex|ps]}). This document may be available within
your TeX installation, but can also be obtained from CTAN, see
page~\pageref{CTAN}.

Another online resource is `Using Imported Graphics in LaTeX2e' by
Keith Reckdahl (file \texttt{epslatex}, also available from CTAN).
Be aware that you cannot blindly trust the information there to be
valid with the Kluwer classes.

\subsection{Graphics file formats}

In general, the preferred graphics are tiff\index{tiff graphics
format} for images/bitmaps and Encapsulated
PostScript\index{Encapsulated PostScript}, eps\index{eps graphics
format} in short, for any type of graphic, including bitmaps. Our
TeX installation requires eps, but we can easily convert tiff to
eps. Many other formats, e.g. pict (Macintosh), wmf (Windows) and
various proprietary formats, are not suitable. Even if we can read
such files, there is no guarantee that they will look the same on
our systems as on yours. Don't use bm2font either.

Simple line art without gray objects can easily be scanned by us, so
as long as you provide good hardcopy of such pictures there will be
no problem.

For bitmapped graphics, the following resolutions are optimal:
black-and-white line figures: 600--1200 dpi; line figures with some
grey or coloured lines: 600 dpi; photographs: 300 dpi; screen shots:
leave as is; any resampling at all will compromise the quality of
screenshots. These are resolutions \emph{after} scaling. Higher
resolutions won't improve output quality but do increase file size
and increase the likelihood of problems at print time; lower
resolutions may compromise output quality.

If your program does not export to eps, you can let the printer
driver generate an eps file; see below.

\subsection{Setting up your Windows printerdriver for eps
generation}\index{Eps generation}

Create a printer entry especially for this purpose. Pick a
printermodel without non-standard features. When available,
`Distiller' is a good choice; an alternative is `Apple Laserwriter
Plus'. Specify `FILE:' as printer port. Then each time you send
something to the `printer' you will be asked for a filename. This
will be the name of your EPS file.

The Encapsulated PostScript (EPS) export option can be found under
the PostScript tab. EPS export should be used only for single pages.

If your PostScript driver doesn't have this option (Windows
NT)\index{Windows NT} then we can probably use an ordinary
PostScript dump. Select a PostScript option such as `Optimize for
portability' or `Archive'.

\subsection{Eps problems}\index{Eps problems}

Eps files are often broken. Frequent problems are: very thin lines
(thinner than, say, 0.25pt, or even thickness 0), very light tints
(less than 10\% gray), exotic fonts and invalid PostScript. Use
standard fonts (Times, Helvetica, Symbol) as much as possible.
Convert text to outlines if you know how to. Test your files by trying
to print them.

Even so, printing to an imagesetter is something very different from
printing to a desktop printer, even if it is a PostScript printer.
So cross your fingers and supply good hardcopy. That way, at least
we can spot problems.

%%%%%%%%%%%%%%%%%%%%%%%%%%%%%%%%%%%%%%%%%%%%%%%%%%%%%%%%%%%%%%%%%%%%

\section{References}\index{References}
{\sloppy
References are inserted in the
\texttt{thebibliography}\Eindex{thebibliography} environment. We
distinguish two styles of References: by name (explained below
in section~\ref{alpharef}) and numbered (explained in
section~\ref{numref}). You can select one of these styles by
including the appropriate document option in the
\verb+\documentclass+ command. \emph{Please refer to the style of
the journal concerned when choosing an option for your references.}
If no document option is included, the class file will usually
presume named references.  \par}

\subsection{References by name}
\label{alpharef}\index{References!alphabetical}

{\sloppy
For references by name one should include the
document option \texttt{namedreferences}\Oindex{namedreferences}
in the documentclass command:\par}

\begin{verbatim}
\documentclass[namedreferences]{kluwer}
\end{verbatim}

{\sloppy \noindent For the \verb+\bibitem+ \Cindex{bibitem} entries
there are some special, non-standard formatting requirements. You
are urged to use Bibtex with the
\texttt{klunamed}\Cindex{klunamed.bst} bibliography style, see
section~\ref{bibtex}, which handles these requirements
automatically. All Kluwer journals which use named references accept
the \texttt{klunamed} bibliography style.

Now about these requirements: every \verb+\bibitem+ in the
\texttt{thebibliography}\Eindex{thebibliography} environment should
have a \verb+\citeauthoryear+\Cindex{citeauthoryear} command as an
option and should have a key to refer to. The \verb+\citeauthoryear+
should have two arguments: the author's name(s) and the year. The
\verb+\citeauthoryear+ should be preceded by a
\verb+\protect+\Cindex{protect} and in the case of more than two
authors the argument with the author's name(s) should contain the
first author's name with an et al., \emph{e.g.}:\par}

\begin{verbatim}
\bibitem[\protect\citeauthoryear{authorname(s)}{year}]{key}
\end{verbatim}

\noindent The first part of your References section could look like this:
\Eindex{thebibliography}

\bigskip
{\noindent\footnotesize\verb+\begin{thebibliography}{} %+
\quad Note the empty pair of curly braces!
\bigskip
\begin{verbatim}
\bibitem[\protect\citeauthoryear{Smith et al.}{1992}]{Smith92}
Smith, I.N., Johnes, R.S. and Hines, W.P.: 1992,
`Title of the Article',
{\it Journal\/} {\bf Vol.~no.~X},
pp.~00--00
\end{verbatim}
}

{\sloppy\noindent
The following \verb+\cite+ 
\Cindex{cite}
\Cindex{shortcite}
\Cindex{citeauthor}
\Cindex{citeyear}
\Cindex{inlinecite}
\Cindex{opencite}
commands can now be used in
your article: \verb+\cite+, \verb+\shortcite+, \verb+\citeauthor+,
\verb+\citeyear+, \verb+\inlinecite+ and \verb+\opencite+.  For the
sample \verb+\bibitem+ above, the four {\tt cite} commands would
generate the following reference in your text:\par}

\medskip
\begin{tabular}{ll} \hline
Call:                &Produces:\\ \hline
\verb+\cite{Smith92}+       &\cite{Smith92}\\
\verb+\citeauthor{Smith92}+ &\citeauthor{Smith92}\\
\verb+\citeyear{Smith92}+   &\citeyear{Smith92}\\
\verb+\shortcite{Smith92}+  &\shortcite{Smith92}\\
\verb+\inlinecite{Smith92}+ &\inlinecite{Smith92}\\
\verb+\opencite{Smith92}+   &\opencite{Smith92} \\ \hline
\end{tabular}
\medskip


\noindent With these commands or combinations of them you can create
any reference in your text according to the reference style of the
journal in which your article is to be published.

If you don't want to make use of
references, but just want to generate a bibliographic
\index{References!typesetting a sorted list}
list at the end of your article, you can
use empty square brackets and braces, like this:
\verb+\bibitem[]{}+.\Cindex{bibitem}

You can format the bibitems manually, but it is better to let Bibtex
generate them, using the \texttt{klunamed} bibliography style; see
page~\pageref{bibtex}.  For the layout conventions we refer you to
the authors instructions of the journal in which your article is to
be published.

\subsection{Numbered References}
\label{numref}\index{References!numbered}

{\sloppy
For numbered references one should include the documentclass option
\texttt{numreferences}\Oindex{numreferences} in the preamble
command: \par}

\begin{verbatim}
\documentclass[numreferences]{kluwer}
\end{verbatim}

\Eindex{thebibliography}
\noindent You should start the \texttt{thebibliography} environment
with the command:

\begin{verbatim}
\begin{thebibliography}{99}
\end{verbatim}

\noindent The second pair of curly braces should contain a number
that corresponds with the widest number in the reference list below
(usually 99 is the correct value).

{\sloppy
\Cindex{bibitem}
\Cindex{cite}
You should label every \verb+\bibitem+ by putting a key between the
curly braces, e.g.: \verb+\bibitem{Smith92}+
In your article you can refer to this
\verb+\bibitem+ with \verb+\cite{Smith92}+.\par}

\subsubsection{Using the optional argument of {\tt bibitem}}

{\sloppy
With numbered references, you should not use the optional
argument to \verb+\bibitem+, since it will produce
unexpected results.\par}

The \verb+\cite+ command will produce the number of the reference in
brackets, e.g. [2]. The \verb+\shortcite+, \verb+\citeauthor+ ...
commands, as mentioned in
Section~\ref{alpharef}, can also be used but they will produce the same
output as \verb+\cite+.

You can either format the bibitems manually or let Bibtex
generate them; see below.  For the layout conventions we refer you
to the authors instructions of the journal in which your article is
to be published.

\subsubsection{Customization}
For numbered references, brackets around both bibliographic
references and citations are now parameterized: if e.g. you want
brackets around the numbers of the bibliographic entries but no dot
after the number then you can put the following definitions is the
preamble:
\begin{verbatim}
\def\redot{}
\def\releft{[}
\def\reright{]}
\end{verbatim}
In the same vein, brackets around citations in the text are defined
by \verb+\def\coleft{[}+ and \verb+\def\ccright{]}+.

\subsubsection{Ranges of references}

If you want to collapse a range of citations such as [3,4,5] into
something like [3--5] then you can load the package
\texttt{klucite}:
\begin{verbatim}
\usepackage{klucite}
\end{verbatim}
It only works if the references are in the correct order.

\subsection{\BibTeX}
\label{bibtex}\index{Bibtex}\index{References!using bibtex}
\index{Using bibtex}\index{Bibliography|see{References}}

If you use \BibTeX, please send the {\tt bbl} file, not the
\texttt{.bib} files. You could also include the \BibTeX\ output
directly in your article.

Kluwer has \BibTeX\ style files available for the
\texttt{namedreferences}\Oindex{namedreferences} and
\texttt{numreferences}\Oindex{numreferences} options. For named
references, the file is called
\texttt{klunamed.bst}\index{Bibtex!klunamed.bst}. You are strongly
urged to use this one because of the special requirements for named
references. For \texttt{numreferences}\Oindex{numreferences}, the
file is called \texttt{klunum.bst}\index{Bibtex!klunum.bst}. You can
substitute other bibliography styles for numbered references if you
want. Both files are included with the distribution of the class
file.

We refer you to the \LaTeX\ and \BibTeX\ manuals for further
information on using \BibTeX.

%%%%%%%%%%%%%%%%%%%%%%%%%%%%%%%%%%%%%%%%%%%%%%%%%%%%%%%%%%%%%%%%%%%%

\section{Various}



\subsection{My pages come out raggedbottom!}

The\index{short pages}
sectioning commands always try to keep a section head together with
a certain number of subsequent lines on
the same page. Values may differ, but the
typical values are: 3 lines for section and subsection and 2 lines
for subsubsection.

If you find that your pages come out unbalanced,
this could be the problem. The other likely problem is
that you used a \texttt{H}ere float on the page, and it didn't fit.

\subsection{AMSmath and cross-referencing}

If you use AMSmath,\index{AMSmath} you may have trouble with
cross-referencing your equations. You can solve this by omitting the
\verb+\begin/end{article}+ lines.

The \texttt{article} environment is designed to allow multiple
articles in one document. In order to guarantee unique
cross-referencing labels, it internally prepends the article number
as a Roman numeral to cross-referencing labels. We saw no way to
make this work with AMSmath.  However, without an article
environment nothing will be prepended, and cross-referencing will
work o.k.

\subsection{Error messages \emph{Command . . . already defined}}

You can usually ignore messages about commands already being
defined.

\subsection{Bold math}\Pindex{bm}\index{Bold math}

If you fail to get bold symbols, try loading the \texttt{bm} package:
\begin{verbatim}
\usepackage{klups]
...
$y = \bm{\alpha}$
\end{verbatim}
This package should be loaded \emph{after} packages such as klups or
mathtime, if you use such a package.

\subsection{Vertical space with theorems, proofs and display math}

The proof environment is designed to maintain a fixed vertical space
from a preceding theorem environment. If your theorem ends with
displaymath you may not be happy with the effect. In that case,
precede the proof environment with \verb+\vskip0pt+.

\subsection{Authors' instructions}\index{authors' instructions}

You may still encounter outdated authors' instructions for certain
journals.  The instructions may require
double line spacing\index{double line spacing} for submissions, or
state that
the journal does not accept electronic
figures\index{Figures!electronic}. Neither of these statements
applies to \LaTeX\ submissions. You do not have to change the
baselines and we are happy with Encapsulated Postscript
figures.\index{Encapsulated
PostScript}\index{Figures!EPS}

\subsection{Double abstracts}

Sorry,\index{multiple abstracts}
but you cannot have multiple abstracts in one article.
If you need that, you have to make one
abstract environment look like two.

\subsection{Letterspaced text, problems with sections and captions}

{\sloppy\noindent
Certain journals require certain headings in a letterspaced font.
While the stylefile provides for this, it interferes with writing to
auxiliary files like the Table of Contents files. If you encounter
incomprehensible error messages, the first thing you should try is
to replace the offending \verb+\caption{...}+ or
\verb+\section{...}+ command by \verb+\caption[...]{...}+ or
\verb+\section[...]{...}+. You will sometimes also have to delete
the \texttt{<jobname>.aux} file.

If you tried this, and it didn't work, you can ask support from
\texttt{texhelp@wkap.nl}. Please provide us with a log file that
contains the error you received.\par}

\subsection{Kluwer styles for camera-ready books}

At our website, \texttt{www.wkap.nl}, LaTeX class files are available
for camera-ready books (proceedings, edited volumes and monographs).
These are not based on the journal styles documented here,
and are maintained and supported by our Boston branch, email
\texttt{dthelp@\dbr wkap.\dbr com}.

Some ongoing book projects use different class files which
\emph{are} based on the journal styles and are supported by us.
These class files are being phased out and are available only upon
request. They come with some additional documentation.

\subsection{Obtaining extra fonts / macro packages}

You may want to get extra packages and/or fonts (like the
\texttt{wasy} symbol font) if those are not  provided at your site.

All \TeX\ related files on the internet are available
from a large central archive called CTAN. These are the main
CTAN sites:\label{CTAN}

{\tt
\begin{center}
 ftp.dante.de  (Deutschland) /tex-archive \\
  ctan.tug.org  (TeX Users Group, Boston, U.S.A.) /tex-archive \\
  ftp.tex.ac.uk (England) /tex-archive \\
\end{center}}

{\sloppy
\noindent In the root of the CTAN directory structure you can find a
file \texttt{CTAN.sites} with mirrors. If you decide to download,
please select a site near you (network-wise).  The root of the CTAN
tree also conveniently contains files \texttt{FILES.bydate},
\texttt{FILES.byname} and \texttt{FILES.bysize}.\par}

\bigskip\noindent If you do not have internet access, some of the packages
 from these sites can also be obtained by request from KAP,
 using the address at the end of this document.

\subsection{Submitting bug reports and requests}

{\sloppy
If you have questions regarding our class file, or if you
have discovered a bug, or if you have a request for something
you would like to see included in the next version of this
class file, send you comments to: {\tt texhelp@wkap.nl} (e-mail)
or to KAP PrePress, Achterom 119, 3311 KB Dordrecht, The Netherlands
(regular mail).\par}


\theendnotes

\addcontentsline{toc}{section}{References}

\begin{thebibliography}{}
\bibitem[\protect\citeauthoryear{Goossens \bgroup \em et al.\egroup
  }{1994}]{goossens}
Goossens, M., F. Mittelbach, and A. Samarin:
\newblock {\em The {\LaTeX} Companion}.
\newblock Ad{\-d}i{\-s}on-Wes{\-l}ey, Reading, MA, USA, 1994.

\bibitem[\protect\citeauthoryear{Knuth}{1984a}]{knuth}
Knuth, D.E:
\newblock {\em The {\TeX}book}.
\newblock Ad{\-d}i{\-s}on-Wes{\-l}ey, Reading, MA, USA, 1984.

\bibitem[\protect\citeauthoryear{Lamport}{1985}]{lamport}
Lamport, L.:
\newblock {\em {\LaTeX} -- {A} Document Preparation System -- User's
  Guide and Reference Manual}.
\newblock Ad{\-d}i{\-s}on-Wes{\-l}ey, Reading, MA, USA, 1985.

\bibitem[\protect\citeauthoryear{Smith et al.}{1992}]{Smith92}
Smith, I.N., R.S. Johnes, and W.P. Hines: 1992,
`Title of the Article',
{\it Journal Title in Italics} {\bf Vol.~no.~X},
pp.~00--00

\end{thebibliography}
\printindex
\label{lastpage}
\end{article}
\end{document}
